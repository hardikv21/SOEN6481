%%%%%%%%%%%%  Generated using docx2latex.com  %%%%%%%%%%%%%%

%%%%%%%%%%%%  v2.0.0-beta  %%%%%%%%%%%%%%

\documentclass[12pt]{article}
\usepackage{amsmath}
\usepackage{latexsym}
\usepackage{amsfonts}
\usepackage[normalem]{ulem}
\usepackage{array}
\usepackage{amssymb}
\usepackage{graphicx}
\usepackage[backend=biber,
style=numeric,
sorting=none,
isbn=false,
doi=false,
url=false,
]{biblatex}\addbibresource{bibliography.bib}

\usepackage{subfig}
\usepackage{wrapfig}
\usepackage{wasysym}
\usepackage{enumitem}
\usepackage{adjustbox}
\usepackage{ragged2e}
\usepackage[svgnames,table]{xcolor}
\usepackage{tikz}
\usepackage{longtable}
\usepackage{changepage}
\usepackage{setspace}
\usepackage{hhline}
\usepackage{multicol}
\usepackage{tabto}
\usepackage{float}
\usepackage{multirow}
\usepackage{makecell}
\usepackage{fancyhdr}
\usepackage[toc,page]{appendix}
\usepackage[hidelinks]{hyperref}
\usetikzlibrary{shapes.symbols,shapes.geometric,shadows,arrows.meta}
\tikzset{>={Latex[width=1.5mm,length=2mm]}}
\usepackage{flowchart}\usepackage[paperheight=11.69in,paperwidth=8.27in,left=1.0in,right=1.0in,top=1.0in,bottom=1.0in,headheight=1in]{geometry}
\usepackage[utf8]{inputenc}
\usepackage[T1]{fontenc}
\TabPositions{0.5in,1.0in,1.5in,2.0in,2.5in,3.0in,3.5in,4.0in,4.5in,5.0in,5.5in,6.0in,}

\urlstyle{same}


 %%%%%%%%%%%%  Set Depths for Sections  %%%%%%%%%%%%%%

% 1) Section
% 1.1) SubSection
% 1.1.1) SubSubSection
% 1.1.1.1) Paragraph
% 1.1.1.1.1) Subparagraph


\setcounter{tocdepth}{5}
\setcounter{secnumdepth}{5}


 %%%%%%%%%%%%  Set Depths for Nested Lists created by \begin{enumerate}  %%%%%%%%%%%%%%


\setlistdepth{9}
\renewlist{enumerate}{enumerate}{9}
		\setlist[enumerate,1]{label=\arabic*)}
		\setlist[enumerate,2]{label=\alph*)}
		\setlist[enumerate,3]{label=(\roman*)}
		\setlist[enumerate,4]{label=(\arabic*)}
		\setlist[enumerate,5]{label=(\Alph*)}
		\setlist[enumerate,6]{label=(\Roman*)}
		\setlist[enumerate,7]{label=\arabic*}
		\setlist[enumerate,8]{label=\alph*}
		\setlist[enumerate,9]{label=\roman*}

\renewlist{itemize}{itemize}{9}
		\setlist[itemize]{label=$\cdot$}
		\setlist[itemize,1]{label=\textbullet}
		\setlist[itemize,2]{label=$\circ$}
		\setlist[itemize,3]{label=$\ast$}
		\setlist[itemize,4]{label=$\dagger$}
		\setlist[itemize,5]{label=$\triangleright$}
		\setlist[itemize,6]{label=$\bigstar$}
		\setlist[itemize,7]{label=$\blacklozenge$}
		\setlist[itemize,8]{label=$\prime$}

\setlength{\topsep}{0pt}\setlength{\parskip}{9.96pt}
\setlength{\parindent}{0pt}

 %%%%%%%%%%%%  This sets linespacing (verticle gap between Lines) Default=1 %%%%%%%%%%%%%%


\renewcommand{\arraystretch}{1.3}


%%%%%%%%%%%%%%%%%%%% Document code starts here %%%%%%%%%%%%%%%%%%%%



\begin{document}
{\fontsize{20pt}{24.0pt}\selectfont Context of use model\par}\par


\vspace{\baselineskip}
The context of use framework chosen is user centric that is it is created from the perspective of user.\par

References[Alonso-Ríos, Vázquez-García, Mosqueira-Rey, Moret-Bonillo,2010; Alonso-Ríos, Mosqueira-Rey, Moret-Bonillo, 2018]\par


\vspace{\baselineskip}

\vspace{\baselineskip}


%%%%%%%%%%%%%%%%%%%% Figure/Image No: 1 starts here %%%%%%%%%%%%%%%%%%%%

\begin{figure}[H]
	\begin{Center}
		\includegraphics[width=5.21in,height=2.36in]{./media/image1.jpeg}
	\end{Center}
\end{figure}


%%%%%%%%%%%%%%%%%%%% Figure/Image No: 1 Ends here %%%%%%%%%%%%%%%%%%%%

\par


\vspace{\baselineskip}
 (source: \href{https://www.tandfonline.com/doi/full/10.1080/10447318.2018.1424101}{https://www.tandfonline.com/doi/full/10.1080/10447318.2018.1424101})\par

\textbf{1) USER}\par

1.1) Role\par

the role of the users who wants to transit through STM will be able to recharge their monthly,weekly or daily pass.\par

the user can also book for one time transit and the user can also recharge their STM pass by paying through their debit card or by cash.\par

1.2) Experience\par

the practical skills and knowledge needed to operate the iGO are basic.the user must be acquainted with making of online transactions while recharging or getting a transit ticket from the machine.also the user must be aware how to board the train or a bus by following the intructions,time and directions.\par


\vspace{\baselineskip}

\vspace{\baselineskip}
1.3) Education\par

the language constraints on TVM are french or english so the user must be familier with the two langauges.Since the it is a TVM which is computer and thus user must be familiar with basic operations of computer.\par


\vspace{\baselineskip}

\vspace{\baselineskip}

\vspace{\baselineskip}
1.4) Physical characterstics\par

the iGO will be a kiosk, thus a normal human without any disability who is able to stand can use the machine for physically handicapped,the person on wheelchair will be able to access the machine as it will be installed on an height against the wall which will be suitable for him/her to operate.The people who are blind will not be able to use the machine as there is no facility to provide reading of text on screen.\par


\vspace{\baselineskip}
\textbf{2) TASK}\par

2.1) Choice in system use\par

The user has a choice among racharging his/her card or paying for transit by either using the iGo or going to iGO representative's office to pay cash.\par


\vspace{\baselineskip}
2.2) Complexity\par

The interface of iGo will not be complex for the user if he/she knows the computer basics and knows how to do transactions by debit/credit card.\par


\vspace{\baselineskip}
2.3) Temporal characterstics\par

the task duration for recharging the iGo or buying a transit fare is quick as there is almost negligible processing time.for the iGo, the usage of the machine can be frequent as there can be queue of people waiting to use the machine.\par

2.4) Demands\par

The demands for task completion inculcates of power supply, internet connectivity to process the transactions, the paper to print the receipts and tickets.\par


\vspace{\baselineskip}

\vspace{\baselineskip}
2.5) Workflow controllabity\par

The user has a control on his interaction with iGo, as the user can cancel the transaction anytime by pressing the cancel/annuler button on the keypad and  can start over the transaction again. but once the transaction is made, the recahrge or ticket can not be refunded.\par

2.6) Safety\par

The iGo is really safe to use by the user and also safer for the enviroment too.the interaction with iGo is neutral and does not involve the emotions of the user and does not harm the user physically/mentally.\par


\vspace{\baselineskip}
2.7) Criticality\par

The IGo has the produce a decisive transaction ie either it is successfull or unsuccessfull, there no middle ground result accepted from the machine.\par


\vspace{\baselineskip}

\vspace{\baselineskip}
\textbf{3) ENVIRONMENT}\par

3.1) Physical\par

the iGo does not provide any harm to the physical environment it is in. the device must be used below a roof to protect it from rain water.\par


\vspace{\baselineskip}
3.2) Social\par

the user has to only interact with the iGo system. if he finds any isse then he/she can see the asistant nearby anytime.the interaction between user and iGo does not affect anyone else as the interaction is only between user and iGo system.\par


\vspace{\baselineskip}
3.3) Technical\par

the technical and infrastructure reqd. for the iGo system are:\par

1. a web application developed for the system.\par

2. connectivity with the banks to make the transactions.\par

3. a computer device which can be a tab, a kiosk or a desktop.\par


\vspace{\baselineskip}

\vspace{\baselineskip}
{\fontsize{18pt}{21.6pt}\selectfont \textbf{STAKEHOLDERS}\par}\par


\vspace{\baselineskip}
{\fontsize{14pt}{16.8pt}\selectfont Stakeholders mind map:\par}\par


\vspace{\baselineskip}

\vspace{\baselineskip}


%%%%%%%%%%%%%%%%%%%% Figure/Image No: 2 starts here %%%%%%%%%%%%%%%%%%%%

\begin{figure}[H]
	\begin{Center}
		\includegraphics[width=450.75pt,height=486.75pt]{./media/image2.jpeg}
	\end{Center}
\end{figure}


%%%%%%%%%%%%%%%%%%%% Figure/Image No: 2 Ends here %%%%%%%%%%%%%%%%%%%%


\vspace{\baselineskip}

\vspace{\baselineskip}

\vspace{\baselineskip}
The following are the significant stakeholders for iGo project .[references: Domain modelling introduction:Kamthan]\par


\vspace{\baselineskip}

\vspace{\baselineskip}
Direct users:\par

the direct users are the main consumer of the system iGo as they are the ones who will use the system for transportation and further users can be categorized among registered users(one with iG0 pass), no registered users and differently abled users.all the users which are registered,differently abled should be prioritise as important as they are the main consumer of the product and their satisfaction and feedback is highly important.\par


\vspace{\baselineskip}
Federal government and Quebec government:\par

the\ two governments should be prioritized as important as they maintain the budget for iG0 system and also plans and overview new routes for  iGo.the taxes are collected by both govt.s as collected from the iGo to maintain the budget.\par


\vspace{\baselineskip}

\vspace{\baselineskip}
Business analysts:\par

they have the responsibility to check the iGo data, do the analysis of user data and thus analysing the collections made from the users.\par


\vspace{\baselineskip}
iGo management:\par

the representatives of iGo,managers, staff and operators are conclude as iGo management team and highy important and influential for the product\par


\vspace{\baselineskip}
iGo developrs:\par

the programmers and developers responsible for the development of the system taking into the considerations of design by the designers.\par


\vspace{\baselineskip}

\vspace{\baselineskip}

\vspace{\baselineskip}
iGo interface designers:\par

the designers who has the responsibilty to meet the user satisfaction when they operate the iGo system and design the interface suitable for every type of direct users.\par

iGo testers:\par

the testers are responisble for regularly keeping the product in check and perform various tests to check for abnormalities.\par


\vspace{\baselineskip}
Equipment providers:\par

responsible for providing the machines,paper,monitors, kepyads,payments equipments,etc\par


\vspace{\baselineskip}
Financial institutions:\par

the banks have a significant role as the payments made are deposited into bank with iGo and also during the transactions,debit/credit cards are validated from the banks.\par


\vspace{\baselineskip}
{\fontsize{16pt}{19.2pt}\selectfont \textbf{Negative stakeholders}\par}[references:use\_case\_modeling\_negative,Kamthan]\par

these are the stakeholders which can degrade the overall quality and processing of iGo and also try to harm the any module of the system.\par


\vspace{\baselineskip}
Misusers:\par

there can be users who do not have their own cards but must be using cards of other users and thus enjoy the privileges of iGo.also the misusers can make payments for transit fares by using fake/stolen credit cards.inpectors should be responsible for check the passes of doubtful passengers.\par


\vspace{\baselineskip}

\vspace{\baselineskip}

\vspace{\baselineskip}
Vandals:\par

the people who can destroy the iGo equipments or harm iGo management team. and such cases are not frequently occuring.anyway, security and inpectors should be regularly checking for such cases.\par


\vspace{\baselineskip}

\printbibliography
\end{document}
