\documentclass[12pt]{report}
\usepackage{tikz}
\usepackage{changepage} 
\usepackage{/Program/LaTeX/tex/latex/pgf/frontendlayer/tikz-uml}
\usepackage[section]{placeins}
\usepackage[english]{babel}
\begin{document}

\chapter{Use Case Model}
\section{Use Case Diagram}
The UML Use Case Diagram can provide a high-level graphical representation of the actors, use cases, and their interactions. It also provides structural relationships between use cases, and structural relationships between actors.\\
\\
Based on the advantages of UML Use Case Diagram, this kind of diagram is chosen for modeling the iGo.\\
\\
The following diagram shows the basic use case diagram of iGo:
\begin{figure}[htb]
\centering
\begin{adjustwidth}{-2cm}{}
  \begin{tikzpicture}
    
	\begin{umlsystem}[x=3,y=0]{iGo System}
	\umlusecase[x=0,y=0,width=1cm,name=AT]{Access TVM}
	\umlusecase[x=3,y=2,width=3cm,name=AO]{Access OPUS card service}
	\umlusecase[x=3,y=-2,width=3cm,name=AN]{Access normal service}
	\umlusecase[x=6,y=0,width=2.5cm,name=ST]{Select ticket type}
	\umlusecase[x=10,y=2,width=2cm,name=VI]{View information}
	\umlusecase[x=10,y=0,width=2cm,name=CP]{Choose payment method}
	\umlusecase[x=7,y=-3,width=1cm,name=P1]{Pay by cc}
	\umlusecase[x=9,y=-3,width=1cm,name=P2]{Pay by db}
	\umlusecase[x=11,y=-3,width=1cm,name=P3]{Pay by cash}
	\umlusecase[x=9,y=-5,width=2cm,name=AP]{Authorize payment}
	\umlusecase[x=4.5,y=-4,width=2cm,name=APO]{Add pass on card}
	\umlusecase[x=4.5,y=-6,width=2cm,name=PT]{Print tacket}
	\umlusecase[x=0,y=-5,width=1.5cm,name=PR]{Print receipt}
	\end{umlsystem}
	
	\umlactor[x=0,y=0]{Traveller}
	\umlactor[x=16.5,y=-5]{Bank}
	
	\umlinclude{AT}{AN}
	\umlinclude{AO}{ST}
	\umlinclude{AN}{ST}
	\umlinclude{AO}{VI}
	\umlextend{AO}{AN}
	\umlextend{PR}{PT}
	\umlextend{PR}{APO}
	\umlassoc{ST}{CP}
	\umlassoc{P1}{AP}
	\umlassoc{P2}{AP}
	\umlassoc{P3}{AP}
	\umlassoc{AP}{APO}
	\umlassoc{AP}{PT}
	\umlinherit{P1}{CP}
	\umlinherit{P2}{CP}
	\umlinherit{P3}{CP}
	
	\umlassoc{Traveller}{AT}
	\umlassoc{Bank}{AP}

  \end{tikzpicture}
  \caption{Use case diagram of iGo System (UCMiGo)}
\end{adjustwidth} 
\end{figure} 


\begin{figure}[htb]
  \centering
  \begin{tikzpicture}
  	\umlactor[x=6,y=0]{abstract Traveller}
  	\umlactor[x=3,y=-3]{citizen}
  	\umlactor[x=5,y=-3]{student}
  	\umlactor[x=7,y=-3]{senior}
  	\umlactor[x=9,y=-3]{disabled}
  	\umlinherit{citizen}{abstract Traveller}
  	\umlinherit{student}{abstract Traveller}
  	\umlinherit{senior}{abstract Traveller}
  	\umlinherit{disabled}{abstract Traveller}
  	
  \end{tikzpicture}
  \caption{Detail of abstract traveller actor}
\end{figure}

\section{Use case}
\begin{itemize}
	\item Access normal service\\
	The user access the normal TVM service without any using OPUS card, after successfully purchased, the TVM will give the user a ticket automatically.
	\item Access OPUS card service\\
	The user can also insert an OPUS card, instead of printing a ticket after purchased, the TVM will update the information of the OPUS card.
	\item View information\\
	The user can view information of their OPUS card after insertion, including the valid date of the card and in pass information.
	\item Authorize payment\\
	A secondary actor Bank is also take part in this use case, basically the Bank will check the validity of the card and charge fee based on the chosen ticket type if it pass the validity phase.
\end{itemize}

\section{Sequence Diagram of Making Payment}
\begin{figure}[htb]
\centering
  \begin{tikzpicture}
  \begin{umlseqdiag}
    \umlactor[scale=0.5,no ddots]{Traveller}
    \umlobject[no ddots]{iGo}
    \umlobject[no ddots]{STM}
    \umlobject[no ddots]{Bank}
    \begin{umlcall}[op=check TVM validity,type=asynchron]{Traveller}{iGo}
      \begin{umlcall}[op=give respond,type=asynchron]{iGo}{Traveller}
      \end{umlcall}
    \end{umlcall}
    \begin{umlfragment}
    \begin{umlcall}[op=insert card,type=synchron,return=show card info]{Traveller}{iGo}
      \begin{umlcallself}[op=process,type=asynchron]{iGo}
      \end{umlcallself}
    \end{umlcall}
    \end{umlfragment}

  \end{umlseqdiag}
  \end{tikzpicture}
  \caption{First part of the sequence diagram}
 \end{figure}
 
 \begin{figure}[htb]
   \centering
   \begin{tikzpicture}
     \begin{umlseqdiag}
     \umlactor[scale=0.5,no ddots]{Traveller}
     \umlobject[no ddots]{iGo}
     \umlobject[no ddots]{STM}
     \umlobject[no ddots]{Bank}
      \begin{umlcall}[op=select ticket type,type=asynchron]{Traveller}{iGo}
      \begin{umlcall}[op=select payment method,type=synchron]{Traveller}{iGo}
        \begin{umlcall}[op=send request,type=synchron,return=respond with link]{iGo}{STM}
          \begin{umlcallself}[op=verify,type=asynchron]{STM}
          \end{umlcallself}
        \end{umlcall}
        \begin{umlcall}[op=respond with link,type=return]{iGo}{Traveller}
        \end{umlcall}
        \begin{umlcall}[op=provide payment details,type=synchron,return=return purchase info]{Traveller}{iGo}
          \begin{umlcall}[op=process payment,type=synchron]{iGo}{Bank}
            \begin{umlcall}[op=valid card info,type=asynchron]{Bank}{Bank}
            \end{umlcall}
            \begin{umlfragment}[type=alt,label=info valid]
              \begin{umlcallself}[op=charge fee,type=asynchron]{Bank}
              \end{umlcallself}
              \begin{umlcall}[op=authorized,type=return]{Bank}{STM}
                \begin{umlcallself}[op=update info,type=asynchron]{STM}
                \end{umlcallself}
                \begin{umlcall}[op=update success,type=return]{STM}{iGo}
                \end{umlcall}
              \end{umlcall}
              \umlfpart[else]
              \begin{umlcall}[op=reject,type=return]{Bank}{STM}
                \begin{umlcall}[op=update fail,type=return]{STM}{iGo}
                \end{umlcall}
              \end{umlcall}
            \end{umlfragment}
          \end{umlcall}
        \end{umlcall}
      \end{umlcall}
    \end{umlcall}
     \end{umlseqdiag}
   \end{tikzpicture}
   \caption{Second part of the sequence diagram}
 \end{figure}
\section{Positive Scenarios}
\begin{figure}[htb]
  \centering
  \begin{tikzpicture}
    \tikzset{start/.style={circle,minimum width=0.3cm ,minimum height=0.3cm, draw ,fill}}
    \tikzset{activity/.style={rectangle,minimum width=1cm ,minimum height=0.5cm,rounded corners =5pt,draw}}
    \tikzset{end/.style ={draw ,double =white , circle ,inner sep =4pt , minimum width =0.3 cm , minimum height =0.3cm,draw,fill}}
    \tikzset{decision/.style ={diamond,minimum width=0.2cm ,minimum height=0.2cm , draw}}
    \draw (0,0) node[start](start){};
    \draw (0,-1.5) node[activity](a1){Access TVM};
    \draw (0,-4) node[decision](d1){Have card?};
    \draw (-2.5,-6.5) node[activity](a2){Access card service};
    \draw (2.5,-6.5) node[activity](a3){Access normal service};
    \draw (-2.5,-8) node[activity](a4){Choose and pay};
    \draw (2.5,-8) node[activity](a5){Choose and pay};
    \draw (-2.5,-10) node[decision](d2){Success?};
    \draw (2.5,-10) node[decision](d3){Success?};
    \draw (0,-11.5) node[activity](a6){Purchase fail};
    \draw (0,-13) node[end](e1){};
    \draw (-6,-11.5) node[activity](a7){Purchase success};
    \draw (6,-11.5) node[activity](a8){Purchase success};
    \draw (-6,-13) node[activity](a9){Add to card};
    \draw (6,-13) node[activity](a10){Print ticket};
    \draw[->](start) -- (a1);
    \draw[->](a1) -- (d1);
    \draw[->](d1) -- node[left]{yes}(a2);
    \draw[->](d1) -- node[right]{no}(a3);
    \draw[->](a2) -- (a4);
    \draw[->](a3) -- (a5);
    \draw[->](a4) -- (d2);
    \draw[->](a5) -- (d3);
    \draw[->](d2) -- node[left]{yes}(a7);
    \draw[->](d2) -- node[right]{no}(a6);
    \draw[->](d3) -- node[left]{no}(a6);
    \draw[->](d3) -- node[right]{yes}(a8);
    \draw[->](a6) -- (e1);
    \draw[->](a7) -- (a9);
    \draw[->](a8) -- (a10);
    \draw[->](a9) -- (e1);
    \draw[->](a10) -- (e1);
    
  \end{tikzpicture}
  \caption{Scenarios}
\end{figure}
\end{document}