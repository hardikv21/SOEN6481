\documentclass[12pt]{report}
\usepackage{fullpage}
\usepackage{tikz}
\usepackage{changepage} 
\usepackage[toc,automake,acronym]{glossaries}
\usepackage[section]{placeins}
\usepackage[english]{babel}
\usepackage[utf8]{inputenc}

\title{Software Requirement Specification - Deliverable 1 }
\author{sanjana.concordia} 
\date{October 2019}



\begin{document}

\section{Problem 1 : Brief description of the kind of TVM selected- Société de transport de Montréal (STM)\newline}

A Ticket Vending Machine (TVM) is a kiosk that produces paper based tickets or recharges a subscription card for a particular domain. Here, we are specifically talking about a TVM for transportation services.
\\
\newline
We have chosen the TVM that is used for Société de transport de Montréal (STM) which is the enterprise that provides public bus and metro transportation services in the city of Montreal, Quebec, Canada. 
\\
\newline
This TVM provides the users with options to get:
1)a one time ticket which can be a One or two way pass, 3 days pass, weekend pass.
2)Recharge the Opus card for a weekly, monthly, four-months or an yearly subscription. 
\\
\newline
If a user does not have an Opus card; they select the option of “Non chargeable” and proceed with selecting the type of fare and make a payment with a debit/credit card via a payment gateway. The TVM process the payment and generates a paper based ticket and a receipt for the transaction.
\\
\newline
If a user has an Opus card; they have to put in their Opus card in the card reader and chose the option for “Rechargeable card”. This will give them the option to choose among the possibility of recharge that they can opt for depending what their Opus card is registered under. For example: if you are a student you have the option to recharge your Opus card for an amount of CAD 52 lasting for a month. After that, the payment is made via a payment gate way, the Opus card is recharged and the TVM generates a receipt for the transaction.  
\\
\newline
The TVM used for STM provides service in English as well as French to cover the requirement of a Bilingual city. 
 \\
\newline
Every metro station in Montreal consists of one or more TVMs to make it convenient for the public to get around in the city.

\end{document}